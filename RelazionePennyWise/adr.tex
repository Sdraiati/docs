\section{Analisi dei requisiti}

\subsection{Target}

\textit{Penny Wise} è un servizio dedicato alla gestione delle finanze personali. Il target principale è costituito da individui di età compresa tra i 20 e i 40 anni, che necessitano di strumenti semplici per monitorare le proprie spese. Il servizio non include funzionalità complesse come il calcolo delle tasse o analisi avanzate sui dati, rendendolo meno adatto a un pubblico più anziano o esperto.\\

Inoltre, la funzionalità di condivisione delle spese tra più utenti è ideale per progetti informali, come organizzare viaggi o tenere traccia delle spese comuni in un appartamento. Il target è vario, e le scelte comunicative, strutturali e grafiche del servizio sono orientate verso l'intuitività e la facilità d'uso, permettendo sia ai visitatori occasionali di esplorare il sito con facilità, sia agli utenti esperti di trovare rapidamente le informazioni desiderate.

\subsection{Attori}

Gli attori principali dell'applicazione sono i seguenti:

\begin{itemize}
    \item \textbf{Visitatore}: un utente generico, che non è riconosciuto dal sistema;

    \item \textbf{Utente Registrato}: un utente che ha effettuato il login, per cui è riconosciuto dal sistema.
\end{itemize}

\subsection{Funzionalità}

\textit{Penny Wise} offre le seguenti funzionalità:
\begin{enumerate}
    \item Visualizzazione delle informazioni generali dall'applicativo; disponibile al Visitatore e all'Utente Registrato;

    \item Descrizione della storia che ha condotto allo sviluppo del progetto e 
		introduzione dell persone che lo hanno sviluppato; disponibile al 
		Visitatore e all'Utente Registrato;

    \item Registrazione: creazione di un account valido all'interno del sistema e riconoscimento dell'utente da parte del sistema; disponibile al Visitatore;

    \item Autenticazione: riconoscimento dell'utente da parte del sistema; disponibile al Visitatore che ha le credenziali di accesso;

    \item Modifica delle credenziali dell'account: modifica di nome utente, email e/o password; disponibile all'Utente Registrato;

    \item Creazione di un nuovo progetto: creazione di un progetto, ovvero di un contenitore di spese che è caratterizzato da un nome e da una descrizione; disponibile all'Utente Registrato;

    \item Modifica di un progetto: modifica del nome e della descrizione del progetto; disponibile all'Utente Registrato che ha il progetto;

    \item Eliminazione di un progetto: rimozione del progetto e di tutte le informazioni ad esso correlate dal sistema; disponibile all'Utente Registrato che ha creato il progetto;

    \item Condivisione di un progetto: il sistema fornisce un link speciale, che permette a chiunque di visualizzare il progetto in questione; disponibile all'Utente Registrato che ha il progetto;

    \item Partecipazione ad un progetto: aggiunta del progetto di un altro Utente Registrato tra i propri progetti; disponibile all'Utente Registrato che ha il link di condivisione del progetto;

    \item Dissociazione da un progetto: rimozione del progetto di un altro Utente Registrato dai propri progetti; disponibile all'Utente Registrato che ha un il progetto;

    \item Visualizzazione delle transazioni di un progetto; disponibile
		all'Utente Registrato per tutti i suoi progetti;

    \item Visualizzazione dell'andamento delle spese di un progetto;
    disponibile all'Utente Registrato per tutti i suoi progetti;

    \item Visualizzazione di un grafico a torta di spartizione delle spese per
		tag;  disponibile all'Utente Registrato per tutti i suoi progetti;

    \item Visualizzazione dei partecipanti ad un progetto: ovvero la lista dei nomi utente degli Utenti Registrati che hanno il progetto;
    disponibile all'Utente Registrato per tutti i suoi progetti;

    \item Visualizzazione dei tag di un progetto: la lista dei tag che sono
		assegnabili alle transazioni; disponibile all'Utente Registrato per
		tutti i suoi progetti;

    \item Filtraggio delle spese per periodo; disponibile all'Utente Registrato
		per tutti i suoi progetti;

    \item Filtraggio delle spese per tag; disponibile all'Utente Registrato per
		tutti i suoi progetti;
     
    \item Aggiunta di una transazione ad un progetto; disponibile all'Utente Registrato che ha il progetto;

    \item Modifica di una transazione di un progetto; disponibile all'Utente Registrato che ha il progetto;

    \item Eliminazione di una transazione da un progetto; disponibile all'Utente Registrato che ha il progetto;

    \item Aggiunta di un tag in un progetto: i tag sono abbinabili alle spese, per categorizzarle; disponibile all'Utente Registrato che ha il progetto;

    \item Modifica di un tag di un progetto; disponibile all'Utente Registrato che ha il progetto;

    \item Eliminazione di un tag da un progetto; disponibile all'Utente Registrato che ha il progetto;
\end{enumerate}

\subsection{SEO}

Innanzitutto, abbiamo immaginao il \textit{search intent} dei nostri utenti. 
Abbiamo identificato le seguenti \textit{query} che potrebbero essere 
utilizzate per cercare il nostro sito:
\begin{itemize}
	\item \textit{Penny Wise};
	\item \textit{Gestione spese personali};
	\item \textit{Applicazione per gestire le spese};
	\item \textit{Gestire le spese condivise};
\end{itemize}

A partire da queste \textit{query}, abbiamo scritto i tag \texttt{<title>},
\texttt{<meta name="description">} e \texttt{<meta name="keywords">} delle
pagine del nostro sito. Si noti, che queste informazioni non sono aggiunte alle
pagine che richiedono l'autenticazione, in quanto non sono indicizzate dai
motori di ricerca e non sono accessibili ai visitatori non autenticati.\\
Inoltre, sono state adottate soluzioni tecniche, come la
divisione tra la struttura, la presentazione ed il comportamento per ridurre il
peso delle pagine e migliorare il \textit{ranking} nei motori di ricerca. Le
soluzioni tecniche adottate sono spiegate nel dettaglio nelle sezioni dedicate.
