\documentclass[12pt,a4paper]{article}
\usepackage{amsmath,amsthm,amsfonts,amssymb,amscd}
\usepackage[T1]{fontenc}
\usepackage{amsfonts}

\usepackage{graphicx}           
% Enhanced support for images
\usepackage{float}              
% Improved interface for floating objects
\usepackage{booktabs}           
% Publication quality tables
\usepackage{xcolor}             
% Driver-independent color extensions
\usepackage[top=3cm, bottom=2.5cm, left=2.5cm, right=2.5cm,
	headsep=0.5cm, headheight=2cm, footskip=1.5cm]{geometry}
% Customize document dimensions
\usepackage{comment}            
% Commenting
\usepackage{adjustbox} 
\usepackage{listings}           
% Typeset programs (programming code) within LaTeX
\usepackage{lastpage}           
% Reference last page for Page N of M type footers.
\usepackage{fancyhdr}           
% Control of page headers and footers
\usepackage{hyperref}           
% Cross-referencing 
\usepackage[small,bf]{caption}  
% Captions
\usepackage{multicol}
\usepackage{tikz}               
% Creating graphic elements
\usepackage{circuitikz}         
% Creating circuits
\usepackage{verbatim}          
% Print exactly what you type in
\usepackage{cite}               
% Citation
\usepackage[us]{datetime} 
% Various time format
\usepackage{blindtext}
% Generate blind text
\usepackage[utf8]{inputenc}
\usepackage{array}
\usepackage{makecell}
\usepackage{tabularx}
\usepackage{titlesec}
\usepackage[italian]{babel}
\usepackage{pgfplots}

\linespread{1.5}

\hypersetup{
    draft=false,
    final=true,
    colorlinks=true,
    citecolor=UM_DarkBlue,
    anchorcolor=yellow,
    linkcolor=UM_DarkBlue,
    urlcolor=UM_DarkBlue,
    filecolor=green,      
    pdfpagemode=FullScreen,
    bookmarksopen=false
    }
    
%%%%%%%%%%%%%%%%%%%%%%%%%%%%%%%%%%%%%%%%%%%%%%%%%%%%%%%%%%%%%%
%\lstdefinestyle{Fortran}{
%basicstyle=\scriptsize,        % the size of the fonts that are used for the code
%  breakatwhitespace=false,         % sets if automatic breaks should only happen at whitespace
%  breaklines=false,                 % sets automatic line breaking
%  captionpos=b,                    % sets the caption-position to bottom
%  commentstyle=\color{mygreen},    % comment style
%  extendedchars=true,              % lets you use non-ASCII characters; for 8-bits encodings only, does not work with UTF-8
%  keepspaces=true,                 % keeps spaces in text, useful for keeping indentation of code (possibly needs columns=flexible)
%  keywordstyle=\color{blue},       % keyword style
%  language=[95]Fortran,                 % the language of the code
%  numbers=left,                    % where to put the line-numbers; possible values are (none, left, right)
%  numbersep=5pt,                   % how far the line-numbers are from the code
%  numberstyle=\tiny\color{mygray}, % the style that is used for the line-numbers
%  rulecolor=\color{black},         % if not set, the frame-color may be changed on line-breaks within not-black text (e.g. comments (green here))
%  showspaces=false,                % show spaces everywhere adding particular underscores; it overrides 'showstringspaces'
%  showstringspaces=false,          % underline spaces within strings only
%  showtabs=false,                  % show tabs within strings adding particular underscores
%  stepnumber=1,                    % the step between two line-numbers. If it's 1, each line will be numbered
%  stringstyle=\color{mymauve},     % string literal style
%  tabsize=4,                       % sets default tabsize to 2 spaces
%  title=\lstname                   % show the filename of files
%}
%
%%%%%%%%%%%%%%%%%%%%%%%%%%%%%%%%%%%%%%%%%%%%%%%%%%%%%%%%%%%%%%%%
\definecolor{UM_Brown}{HTML}{3D190D}
\definecolor{UM_DarkBlue}{HTML}{2264B0}
\definecolor{UM_LightBlue}{HTML}{1CA9E1}
\definecolor{UM_Orange}{HTML}{fEB415}

\begin{document}
\pagestyle{empty}
\begin{titlepage}
\noindent\makebox[\linewidth]{
\includegraphics[height=6.5cm]{figures/logo}
}
\vspace{15pt}
\textcolor{UM_Brown}{
\begin{flushleft}
    \textbf{\huge{Relazione Progetto Tecnologie Web}}\\
    \vspace{15pt}
    \huge \textbf{Grigo Verde} \\
    \vspace{15pt}
	\begin{normalsize}
    \url{http://tecweb.studenti.math.unipd.it/scaregna/}
	\newline
    Referente: simone.caregnato@studenti.unipd.it
	\newline
	Credenziali di acesso:
		\begin{itemize}
            \item Utente docente:
            \begin{itemize}
                \item[] Username: user
                \item[] Password: user
            \end{itemize}
            \item Utente amministratore:
            \begin{itemize}
                \item[] Username: admin
                \item[] Password: admin
            \end{itemize}
		\end{itemize}
	\end{normalsize}
\end{flushleft}
}
\vspace{40pt}
\textcolor{UM_Brown}{
\begin{flushright}
\begin{scriptsize}
\begin{tabular}{lcl}
    Basso Leonardo & - & 2042329 \\
    Caregnato Simone & - & 2042884 \\
    Igbinedion Osamwonyi Eghosa Matteo & - & 2042888 \\
    Rosso Carlo & - & 2034293 \\
\end{tabular}
\end{scriptsize}
\end{flushright}
\hrule
}
\end{titlepage}


\newpage
\begin{abstract}

\textit{Grigo Verde} è un progetto della scuola Michelangelo Grigoletti di
Pordenone, basata sull'idea dell'insegnante Andrea Rosso. Il progetto vuole
incrementare l'utilizzo delle aree verdi della scuola, in particolare il
giardino, per favorire l'apprendimento e il benessere degli studenti.\\
In particolare, per quanto riguarda il sito web, l'obiettivo è quello di
fornire un'interfaccia semplice e intuitiva per la gestione delle prenotazioni
delle aree verdi della scuola, permettendo ai docenti di organizzare le lezioni
all'aperto evitando sovrapposizioni e garantendo che lo spazio verde prenotato
sia in ombra.

\end{abstract}

\newpage
\tableofcontents
\newpage
\pagestyle{plain}
\pagenumbering{arabic}
\section{Analisi dei requisiti}

\subsection{Target}

\textit{Grigo Verde} è un servizio dedicato alla gestione delle prenotazioni
delle aree verdi della scuola Michelangelo Grigoletti di Pordenone. Il target
principale è costituito dagli insegnanti della scuola, che necessitano di
strumenti semplici per organizzare le lezioni all'aperto.\\
Il servizio è progettato per essere intuitivo e facile da usare, in modo
da permettere agli utenti di prenotare le aree verdi in pochi passaggi. Inoltre,
il servizio vuole essere accessibile da dispositivi mobili, in
modo da permettere agli utenti di prenotare le aree verdi anche in mobilità.
Considerando il target principale, il servizio è progettato per essere
facilmente accessibile e utilizzabile da utenti non esperti di tecnologia.

\subsection{Attori}

Gli attori principali dell'applicazione sono i seguenti:

\begin{itemize}
	\item \textbf{Visitatore}: un utente generico, che non è riconosciuto dal
	      sistema;

	\item \textbf{Utente Docente}: un docente autorizzato, registrato nel
	      sistema e che ha effettuato il login;

	\item \textbf{Utente Amministratore}: un amministratore autorizzato,
	      registrato nel sistema e che ha effettuato il login.
\end{itemize}

\subsection{Funzionalità}

\textit{Grigo Verde} offre le seguenti funzionalità:
\begin{enumerate}
	\item \textbf{Autenticazione}: login tramite username e password e quindi
	      riconoscimento dell'utente da parte del sistema; Disponibile a:
	      \begin{itemize}
		      \item Visitatore;
	      \end{itemize}

	\item \textbf{Logout}: Disponibile a:
	      \begin{itemize}
		      \item Utente docente;
		      \item Utente amministratore;
	      \end{itemize}

	\item \textbf{Visualizzazione pagina informativa sull'applicazione}, in cui
	      viene spiegato lo scopo, le motivazioni delle sua creazione e cosa
	      offre;
	      Disponibile a:
	      \begin{itemize}
		      \item Visitatore;
		      \item Utente docente;
		      \item Utente amministratore;
	      \end{itemize}

	\item \textbf{Visualizzazione degli spazi registrati nel sistema}, con foto
	      descrizione, numero di posti; Disponibile a:
	      \begin{itemize}
		      \item Visitatore;
		      \item Utente docente;
		      \item Utente amministratore;
	      \end{itemize}

	\item \textbf{Filtraggio degli spazi per tipo}, ad esempio per aule o per
	      aree ricreative; Disponibile a:
	      \begin{itemize}
		      \item Visitatore;
		      \item Utente docente;
		      \item Utente amministratore;
	      \end{itemize}

	\item \textbf{Filtraggio degli spazi per disponibilità}, tramite selezione
	      del giorno e dell'intervallo temporale; Disponibile a:
	      \begin{itemize}
		      \item Visitatore;
		      \item Utente docente;
		      \item Utente amministratore;
	      \end{itemize}

	\item \textbf{Visualizzazione delle disponibilità orarie di uno spazio}
	      sulla base del mese selezionato; Disponibile a:
	      \begin{itemize}
		      \item Visitatore;
		      \item Utente docente;
		      \item Utente amministratore;
	      \end{itemize}

	\item \textbf{Prenotazione di uno spazio disponibile}, specificando il
	      giorno da calendario e l'orario; Disponibile a:
	      \begin{itemize}
		      \item Utente docente;
	      \end{itemize}

	\item \textbf{Annullamento prenotazione da lui creata in precedenza};
	      Disponibile a:
	      \begin{itemize}
		      \item Utente docente;
	      \end{itemize}

	\item \textbf{Modifica di qualunque prenotazione}: Disponibile a:
	      \begin{itemize}
		      \item Utente amministratore;
	      \end{itemize}

	\item \textbf{Inserimento di prenotazioni periodiche}: inserimento di
	      prenotazioni che si ripetono in un intervallo di tempo; Disponibile a:
	      \begin{itemize}
		      \item Utente amministratore;
	      \end{itemize}

	\item \textbf{Creazione di un nuovo spazio}, con inserimento di immagini,
	      posizione, nome, tipo, descrizione, numero di tavoli; Disponibile a:
	      \begin{itemize}
		      \item Utente amministratore;
	      \end{itemize}

	\item \textbf{Modifica di uno spazio}: inserimento/eliminazione di immagini,
	      modifica posizione, nome, tipo, descrizione, numero di tavoli; Disponibile
	      a:
	      \begin{itemize}
		      \item Utente amministratore;
	      \end{itemize}

	\item \textbf{Eliminazione di uno spazio}: Disponibile a:
	      \begin{itemize}
		      \item Utente amministratore;
	      \end{itemize}

	\item \textbf{Inserimento disponibilità spazio}: inserimento in un
	      calendario settimanale della disponibilità oraria; Disponibile a:
	      \begin{itemize}
		      \item Utente amministratore;
	      \end{itemize}

	\item \textbf{Creazione di un nuovo tipo di spazio}, con specifica del nome
	      e della possibilità o meno di specificare il numero di posti; Disponibile
	      a:
	      \begin{itemize}
		      \item Utente amministratore;
	      \end{itemize}

	\item \textbf{Visualizzazione della lista degli utenti registrati nel
		      sistema}; Disponibile a:
	      \begin{itemize}
		      \item Utente amministratore;
	      \end{itemize}

	\item \textbf{Registrazione nuovo utente}: aggiunta del nuovo utente al
	      sistema con inserimento di nome, cognome, ruolo (docente o amministratore);
	      Disponibile a:
	      \begin{itemize}
		      \item Utente amministratore;
	      \end{itemize}

	\item \textbf{Eliminazione utente}: Disponibile a:
	      \begin{itemize}
		      \item Utente amministratore;
	      \end{itemize}

	\item \textbf{Modifica utente}: modifica di nome, cognome e/o ruolo;
	      Disponibile a:
	      \begin{itemize}
		      \item Utente amministratore;
	      \end{itemize}
\end{enumerate}

\subsection{SEO}

Innanzitutto, abbiamo immaginao il \textit{search intent} dei nostri utenti.
Abbiamo identificato le seguenti \textit{query} che potrebbero essere
utilizzate per cercare il nostro sito:
\begin{itemize}
	\item \textbf{Grigo Verde};
	\item \textbf{Liceo Grigoletti};
	\item \textbf{Liceo Scientifico Pordenone};
	\item \textbf{Spazi verdi};
	\item \textbf{Aule all'aperto};
\end{itemize}

A partire da queste \textit{query}, abbiamo scritto i tag \texttt{<title>},
\texttt{<meta name="description">} e \texttt{<meta name="keywords">} delle
pagine del nostro sito. Si noti, che queste informazioni non sono aggiunte alle
pagine che richiedono l'autenticazione, in quanto non sono indicizzate dai
motori di ricerca e non sono accessibili ai visitatori non autenticati. In
aggiunta, sono state aggiunte keyword specifiche per ogni pagina, in modo da
migliorare il posizionamento nei motori di ricerca.\\
Finalmente, sono state adottate soluzioni tecniche, come la
divisione tra la struttura, la presentazione ed il comportamento per ridurre il
peso delle pagine e migliorare il \textit{ranking} nei motori di ricerca. Le
soluzioni tecniche adottate sono spiegate nel dettaglio nelle sezioni dedicate.

\newpage
\section{Progettazione}

\subsection{Design Persona}

Partendo dalla risorsa consigliata nel corso (\textit{Desining for Emotion} di
Aaron Walter), abbiamo descritto la personalità di partenza per sviluppare il
prodotto:

\begin{itemize}
	\item \textbf{Brand name}: \textit{Grigo Verde}. Il nome è stato scelto per
	      richiamare il nome della scuola abbinato al colore verde, che richiama
	      gli spazi all'aperto e la natura;

	\item \textbf{Overview}: il sito prosegue il lavoro degli studenti del Liceo
	      Grigoletti che hanno partecipato ad un progetto di riqualificazione
	      delle aree verdi della scuola. Per questo motivo il verde è il colore
	      predominante del sito ed è prensente anche nel nome;

	\item \textbf{Brand traits}:
	      \begin{itemize}
		      \item Formale ma non rigido: deve essere adatto ad un contesto
		            scolastico;

		      \item Semplice e banale: considerando il target del sito, è
		            fondamentale che sia facile da usare e che non ci siano
		            elementi che possano confondere l'utente;

		      \item Preciso: deve essere sempre pronto a rispondere ad ogni
		            esigenza dell'utente con la massima accuratezza;

		      \item Accattivante, ma non complesso: la scelta dei colori è
		            stata fatta in parte con il proponente del progetto, un
		            insegnante di arte, per garantire un impatto visivo
		            accattivante, senza però appesantire il sito.
	      \end{itemize}

	\item \textbf{Personality map}:
	      \begin{center}
		      \begin{tikzpicture}
			      \begin{axis}[
					      axis lines = middle,
					      xlabel = { semplice },
					      ylabel = { formale },
					      xmin = -11, xmax = 11,
					      ymin = -11, ymax = 11,
					      xtick = {-10,-5,...,10},
					      ytick = {-10,-5,...,10},
					      xlabel style={at={(ticklabel cs:1)}, anchor=north},
				      ]
				      % template per mostrare il punto all'interno del piano cartesiano.
				      \addplot [
					      color=black,
					      mark=*,
					      only marks,
				      ] coordinates {
						      (10, 3)
					      } node[below] {$P$};
			      \end{axis}
		      \end{tikzpicture}
	      \end{center}

	      Si noti che si tratta più che altro dell'aspettativa che vogliamo
	      raggiungere, non è detto che siamo stati in grado di raggiungere 10 in
	      semplicità. Tuttavia, vogliamo chiarire che la semplicità è il nostro
	      obiettivo principale;

	\item \textbf{Visual lexicon}:
	      \begin{itemize}
		      \item \textbf{Colori}: verde scuro per i titoli, nero per i testi
		            e bianco per lo sfondo. Ogni tanto ci sono delle decorazioni
		            rosse per attirare l'attenzione dell'utente e per richiamare
		            il colore della scuola;

		      \item \textbf{Contorni}: i contorni arrotondati rendono il
		            prodotto più accattivante e diminuiscono il senso di rigidità
		            e il carico cognitivo;

		      \item \textbf{Font}: \textit{Sans-Serif Arial};
	      \end{itemize}

	\item \textbf{Engagement methods}:
	      \begin{itemize}
		      \item \textbf{Design intuitivo}: l'utente deve essere in
		            grado di capire cosa fare senza dover leggere alcun manuale;

		      \item \textbf{Psicologia dei colori}: sono utilizzati dei colori
		            accattivanti che richiamano il verde della natura;

		      \item \textbf{Feedback}: l'utente deve ricevere un feedback
		            ad ogni azione che compie, in modo da rassicurarlo e
		            mantenere il suo interesse, evitando frustrazioni.
	      \end{itemize}
\end{itemize}

\subsection{Palette}

La palette di colori è stata scelta in base alla personalità del brand;
infatti sono stati selezionati con un contrasto elevato tra loro in
modo da garantire una buona leggibilità anche da parte di utenti con deficit
parziale della vista.
Non solo, ci siamo assicurati che colori simili non fossero accostati in modo
da evitare confusione tra di essi. Di seguito evidenziamo la palette di colori.

\begin{itemize}
	\item Sfondo: \#ffffff;
	\item Colore del testo: \#333333;
	\item Colore primario: \#335833;
	\item Colore secondario: \#0000ff;
	\item Colore terziario: \#73dec1;
	\item Colore per gli errori: \#cf0000;
\end{itemize}

\begin{figure}[h]
	\label{fig:palette}
	\centering
	\includegraphics[width=0.8\textwidth]{figures/palette.png}
	\caption{Palette di colori scelta per il sito.}
\end{figure}

In particolare abbiamo prestato attenzione alle seguenti coppie di contrasto:
\begin{itemize}
	\item sfondo e testo;

	\item sfondo e colore primario: il colore primario è usato per l'header,
	      il footer, i link visitati e per i titoli, mentre lo sfondo è usato
	      per il testo nell'header e nel footer e come sfondo nel resto del
	      sito;

	\item sfondo e colore secondario: il colore secondario è usato per i link
	      non visitati, considerando il target di utenza, è stato scelto un
	      blu accesso, uno standard per questo tipo di link;

	\item colore primario e terziario: pensavamo di usare il colore
	      terziario per segnalare i link all'interno dell'header,
	      tuttavia per evitare sovraccarichi visivi, abbiamo optato per
	      usare il colore dello sfondo sia per il testo che per i link
	      all'interno dell'header.
	      Dunque il colore terziario è stato usato per alternare le righe
	      delle tabelle, in modo da facilitarne la lettura;

	\item sfondo e colore per gli errori: il colore per gli errori è usato
	      per evidenziare i form che sono stati compilati in modo improprio e
	      per spiegare in quale modo correggere le informazioni inserite.
	      Questo colore è stato scelto per contrastare con lo sfondo bianco
	      in modo da attirare l'attenzione dell'utente e facilitare la
	      correzione. Questo colore è usato anche in modo decorativo per
	      rendere il sito più accattivante;

	\item colore secondario e terziario: questi due colori si sovrappongono
	      all'iterno delle tabelle, dove sono presenti i link per andare al
	      dettaglio di una prenotazione.
\end{itemize}

Il contrasto più basso tra queste coppie risulta essere tra il colore primario
e il colore terziario, con un rapporto di 5.01:1, che rispetta comunque lo
standard WCAG AA anche per un testo di dimensione inferiore a 17pt.

\subsection{Accessibilità}

L'accessibilità è un indice di qualità del sito, pertanto è stata fin da subito
un proposito imprescindibile che ha guidato la fase di progettazione e le
successive. Di seguito sono riportate le misure adottate per garantire
un'esperienza di utilizzo ottimale per tutti gli utenti.

\subsubsection{Orientamento dell'utente}

Per garantire un'esperienza di utilizzo ottimale e ridurre il disorientamento e
il sovraccarico cognitivo, sono state adottate diverse misure:

\begin{itemize}
	\item \textbf{Breadcrumb}: utilizzo di breadcrumb in ogni pagina per
	      facilitare la navigazione e mantenere l'utente consapevole della
	      propria posizione all'interno del sito;

	\item \textbf{Link circolari}: controllo rigoroso nella costruzione della
	      pagina per evitare la presenza di link circolari che potrebbero
	      confondere l'utente;

	\item \textbf{Link}: tutti i link sono sottolineati e i link non visitati
	      hanno il classico blu, mentre i link visitati sono di colore verde
	      scuro. L'unica eccezione è il logo il logo del sito in alto a sinistra
	      che è un link che riporta alla home. Tuttavia, all'interno del menù è
	      presente il collegamento alla home, quindi non riteniamo che possa
	      creare confusione;

	\item \textbf{Vai al contenuto}: implementazione della funzionalità "vai al
	      contenuto" per migliorare l'accessibilità agli utenti che utilizzano
	      screen reader oppure che navigano dal telefono;

	\item \textbf{Torna su}: aggiunta del pulsante "torna su" alla fine di
	      ogni pagina, che diventa visibile scorrendo verso il basso,
	      facilitando così il ritorno rapido all'inizio della pagina stessa.
	      Il pulsante non è visibile negli schermi grandi;

	\item \textbf{Linguaggio semplice}: abbiamo adottato un linguaggio adatto al
	      target di utenza; per esempio, all'inizio avevamo usato il termine
	      "dashboard" per indicare la homepage di un utente autenticato, tuttavia
	      ci siamo resi conto che poteva creare confusione, quindi abbiamo
	      cambiato il termine in "cruscotto";

	\item \textbf{Alternative testuali}: tutte le immagini hanno un'alternativa
	      testuale che permetta di comunicare il contenuto dell'immagine a chi
	      non è in grado di vederla;

	\item \textbf{Attributo \texttt{lang}}: l'attributo \texttt{lang} è stato
	      aggiunto in ogni pagina per indicare la lingua principale del sito,
	      in modo da permettere agli screen reader di selezionare la voce
	      corretta per la lettura del testo, inoltre è stato aggiunto anche
	      ad ogni parola o frase in lingua straniera;

	\item \textbf{Tag \texttt{time}}: il tag \texttt{time} è stato utilizzato
	      per indicare date e orari, in modo da permettere agli screen reader
	      di leggere correttamente queste informazioni;

	\item \textbf{Tabelle accessibili}: le tabelle sono state progettate in modo
	      da essere accessibili, secondo le indicazioni approfondite durante il
	      corso;

	\item \textbf{Tag \texttt{aria-label} e \texttt{aria-describedby}}: dove
	      necessario, sono stati aggiunti attributi \texttt{aria-label} e
	      \texttt{aria-describedby} per comunicare informazioni aggiuntive agli
	      screen reader;

	\item \textbf{Creazione e rispetto di convenzioni interne al sito}.
\end{itemize}

\subsubsection{Responsive layout}

Il sito è stato progettato per adattarsi a qualsiasi dispositivo, in modo da
garantire un'esperienza di utilizzo ottimale sia da desktop che da mobile. Per
questo motivo, sono stati adottati layout flessibili e fluidi, in modo da
garantire una buona leggibilità e usabilità indipendentemente dalla dimensione
dello schermo o dalle preferenze dell'utente: infatti abbiamo utilizzato solo
unità di misura proporzionali come \texttt{em} oppure \texttt{\%}, tranne che
per la dimensione minima delle schede che mostrano uno spazio, che è stata
impostata a 150px, per garantire una buona visualizzazione delle immagini.\\
I breakpoint sono stati scelti
con cura per garantire una transizione fluida tra i diversi layout e una
buona esperienza di utilizzo su tutti i dispositivi. Sono rispettivamente:
\begin{itemize}
	\item minore di 600px: layout mobile, per telefoni e piccoli schermi;

	\item tra 601px e 900px: layout tablet, per tablet e schermi di dimensioni
	      medie;

	\item tra i 901px e i 1223px: layout desktop, per schermi di dimensioni
	      medie e per i tablet in modalità landscape;

	\item maggiore di 1224px: layout desktop, per schermi di grandi dimensioni.
\end{itemize}

Infine, per la pagina "About us" abbiamo aggiunto un breakpoint a 1024px per
cambiare il layout della pagina in modo da garantire una migliore leggibilità
e evitare che l'utente si stanchi di leggere un testo troppo lungo.

\subsection{Struttura del sito}

Abbiamo cominciato a progettare il sito partendo da un'analisi delle esigenze e
quindi delle funzionalità che il sito deve offrire. Abbiamo quindi definito
l'elenco delle pagine che compongono il sito e abbiamo diviso le funzionalità
all'interno di queste pagine.
Abbiamo individuato funzionalità comuni a più pagine e le abbiamo raggruppate.
Infine abbiamo collegato le pagine tra loro in modo da definire un percorso
di navigazione logico e intuitivo per l'utente.
Finalmente abbiamo definito la struttura organizzativa del sito, ovvero la
mappa del sito: abbiamo deciso di adottare la struttura gerarchica per garantire
maggiore chiarezza e facilità di navigazione all'utente.\\
Per evitare il rischio del sovraccarico cognitivo, abbiamo deciso di includere 4
voci nel menù principale, in modo da garantire all'utente un accesso rapido alle
pagine principali del sito. Se l'utente effettua il login come docente allora
viene aggiunta la voce "cruscotto" al menù; se l'utente effettua il login come
amministratore allora viene aggiunta anche la voce "utenti". In questo modo il
menù del sito ha un numero di voci variabile da 4 a 6. Infine il sito ha una
profondità di 4 livelli; in questo modo l'utente può raggiungere qualsiasi
pagina in massimo 3 click, più due se l'utente deve effettuare il login.

\begin{figure}[h]
	\centering
	\includegraphics[width=0.8\textwidth]{figures/sitemap.png}
	\caption{Struttura organizzativa del sito.}
\end{figure}

Di seguito sono descritte le pagine del sito:

\begin{itemize}
	\item \textbf{Homepage}: in genere è la prima pagina che viene visualizzata
	      quando si accede al sito Grigo Verde. Contiene una breve descrizione
	      del progetto e delle funzionalità disponibili;

	\item \textbf{Login}: pagina di autenticazione al sito;

	\item \textbf{About us}: pagina che contiene informazioni sul progetto in
	      dettagliato, spiegando come è nato e quali sono gli obiettivi;

	\item \textbf{Cruscotto}: pagina di benvenuto per gli utenti autenticati,
	      contiene il riepilogo delle prenotazioni future o in corso dell'utente
	      oltre che alle azioni rapide che l'utente può compiere;

	\item \textbf{Spazi}: pagina di visualizzazione degli spazi verdi della
	      scuola e delle aree ricreative, è possibile filtrare gli spazi;

	      \begin{itemize}
		      \item \textbf{Dettaglio di uno spazio}: sono visualizzate le
		            informazioni relative ad uno spazio, come il nome,
		            l'immagine o anche le prenotazioni e gli orari di apertura
		            dello stesso. Non solo, se l'utente è autenticato, allora
		            può accedeer alla pagina di prenotazione direttamente da
		            qui;

		            \begin{itemize}
			            \item \textbf{Modifica spazio}: pagina per modificare le
			                  informazioni di uno spazio;

			            \item \textbf{Modifica orari di apertura}: pagina per
			                  modificare gli orari di apertura di uno spazio.
		            \end{itemize}

		      \item \textbf{Nuovo Spazio}: un amministratore può accedere a
		            questa pagina per creare un nuovo spazio.
	      \end{itemize}

	\item \textbf{Prenotazioni}: pagina di visualizzazione delle prenotazioni
	      future registrate nel sistema;

	      \begin{itemize}
		      \item \textbf{Dettaglio di una prenotazione}: sono visualizzate le
		            informazioni relative ad una prenotazione, l'orario della
		            prenotazione, lo spazio prenotato e l'utente che ha
		            effettuato la prenotazione. Se l'utente è autenticato e ha
		            effettuato la prenotazione oppure è un amministratore,
		            allora può cancellare o modificare la prenotazione
		            direttamente da qui;

		            \begin{itemize}
			            \item \textbf{Modifica prenotazione}: pagina per
			                  modificare le informazioni di una prenotazione;
		            \end{itemize}

		      \item \textbf{Nuova prenotazione}: si accede a questa pagina per
		            effettuare una nuova prenotazione. Si noti che un
		            riferimento a questa pagina è prensente anche nel dettaglio
		            di uno spazio e nel cruscotto dell'utente autenticato.
	      \end{itemize}

	\item \textbf{Utenti}: pagina di visualizzazione degli utenti registrati nel
	      sistema. Si noti che questa pagina è accessibile solo agli
	      amministratori, lo stesso vale per le pagine qui sotto;

	      \begin{itemize}
		      \item \textbf{Dettaglio di un utente}: sono visualizzate le
		            informazioni relative ad un utente, oltre che alle sue
		            prenotazioni future;

		            \begin{itemize}
			            \item \textbf{Modifica di un utente}: pagina per
			                  modificare le informazioni di un utente compresa
			                  la password.
		            \end{itemize}

		      \item \textbf{Nuovo utente}: si accede a questa pagina per
		            registrare un nuovo utente all'interno del sistema.
	      \end{itemize}
\end{itemize}

\subsection{SEO}

Innanzitutto, abbiamo immaginao il \textit{search intent} dei nostri utenti.
Abbiamo identificato le seguenti \textit{query} che potrebbero essere
utilizzate per cercare il nostro sito:
\begin{itemize}
	\item \textbf{Grigo Verde};
	\item \textbf{Liceo Grigoletti};
	\item \textbf{Liceo Scientifico Pordenone};
	\item \textbf{Spazi verdi};
	\item \textbf{Aule all'aperto};
\end{itemize}

A partire da queste \textit{query}, abbiamo scritto i tag \texttt{<title>},
\texttt{<meta name="description">} e \texttt{<meta name="keywords">} delle
pagine del nostro sito. Si noti, che queste informazioni non sono aggiunte alle
pagine che richiedono l'autenticazione, in quanto non sono indicizzate dai
motori di ricerca e non sono accessibili ai visitatori non autenticati. In
aggiunta, sono state aggiunte keyword specifiche per ogni pagina, in modo da
migliorare il posizionamento nei motori di ricerca.\\
Finalmente, sono state adottate soluzioni tecniche, come la
divisione tra la struttura, la presentazione ed il comportamento per ridurre il
peso delle pagine e migliorare il \textit{ranking} nei motori di ricerca. Le
soluzioni tecniche adottate sono spiegate nel dettaglio nelle sezioni dedicate.

\newpage
\section{Realizzazione}

Per la realizzazione di questo sito web abbiamo deciso di renderizzare tutti gli
elementi lato server, utilizzando PHP per la gestione delle richieste e delle
risposte. In particolare, la cartella \texttt{src} contiene tutti i file che
riguardano il back-end e la struttura gerarchica data dalle cartelle riflette la
divisione in moduli del codice.\\
Dunque ci sono i seguenti moduli:
\begin{itemize}
	\item \textbf{template}: contiene tutti i file html che sono utilizzati per
	      la creazione delle pagine web;

	\item \textbf{page}: contiene le classi php che si occupano di creare una
	      pagina web per interno oppure solo una porzione di essa;

	\item \textbf{model}: contiene le classi php che si occupano di gestire i
	      dati, le operazioni sul database o la logica di business, come ad
	      esempio la gestione delle sessioni, ovvero dei cookie;

	\item \textbf{controller}: contiene le classi php che si occupano di
	      gestire le richieste e le risposte, assemblano i servizi e le pagine
	      per fornire una risposta coerente all'utente;

	\item \textbf{test}: finalmente, test contiene i file php che si occupano di
	      testare le classi php del progetto. Approfondiamo questo modulo nella
	      sezione \ref{sec:test}.
\end{itemize}

\subsection{Page}

In questo modulo viene definita la classe \texttt{Page} che viene così definita:

\begin{lstlisting}[language=PHP]
class Page
{
    protected function getContent($path);
    protected function setTitle($title);
    protected function addKeywords($keywords);
    protected function setNav($nav);
    protected function setBreadcrumb($breadcrumb);
    protected function takeOffCircularReference($content);
    public function setPath($path);
    public function render();
    public function error($message);
}
\end{lstlisting}

Vogliamo portare l'attenzione su alcuni metodi:
\begin{itemize}
	\item \texttt{takeOffCircularReference}: questo metodo si occupa di
	      rimuovere i link circolari presenti nel contenuto della pagina;

	\item \texttt{render}: questo metodo viene sovrascritto dalle classi figlie,
	      di modo da uniformare la creazione di una pagina web, senza dover
	      modificare il layout di base delle pagine.
\end{itemize}

\subsection{Model}

Il model interagisce con il database, per cui la struttura delle classi
rispecchia quella del database, quindi riportiamo solo la struttura del
database, che si evince dal diagramma in figura \ref{fig:Database} e riportiamo
una classe di esempio per mostrare come vengono gestiti i dati.

\begin{figure}[h]
	\centering
	\includegraphics[width=0.9\textwidth]{figures/Database}
	\caption{Diagramma del database.}
	\label{fig:Database}
\end{figure}

Per stabilire la connessione con il database viene utilizzata l'estensione
\texttt{MySQLi} di PHP.\\
Abbiamo deciso di utilizzare la classe \texttt{Utente} come esempio. Notiamo che
viene utilizzata la classe \texttt{Database} per stabilire una connessione con
il database e per assicurarci che la connessione sia unica: infatti questa
classe implementa il pattern \texttt{Singleton}. La classe \texttt{Model} ha un
istanza della classe \texttt{Database} e la utilizza per eseguire le query; la
classe \texttt{Model} definisce i metodi per interagire con il database.
Finalmente, la classe \texttt{Utente} estende la classe \texttt{Model} e
implementa le operazioni CRUD sull'entità \texttt{Utente}.
Infine, notiamo che la classe \texttt{Utente}, quando inserisce un nuovo utente
all'interno del database utilizza la funzione \texttt{password\_hash} della
libreria standard di PHP per criptare la password, in modo tale che essa non sia
salvata in chiaro nel database e che un eventuale attacco non possa recuperare
la password degli utenti.

\begin{figure}[h]
	\centering
	\includegraphics[width=0.5\textwidth]{figures/model}
	\caption{Diagramma delle classi del model}
	\label{fig:model}
\end{figure}

\subsection{Controller}

Finalmente abbiamo implementato il controller, che si occupa di utilizzare i due
moduli spiegati in questa sezione per gestire le richieste dell'utente. In
particolare, una richiesta ad un controller ritorna una pagina web, che viene
creata utilizzando il modulo \texttt{page}. In figura \ref{fig:controller} è
riportato il diagramma delle classi del controller. Notiamo che viene definta
una classe \texttt{Router} che gestisce un array di \texttt{Endpoint}. Un router
si occupa di individuare e delegare ciascuna richiesta che riceve ad uno dei
suoi endpoint. Un endpoint estende la classe \texttt{Endpoint} e implementa il
metodo \texttt{handle}, che si occupa di gestire la richiesta.
\begin{figure}[h]
	\centering
	\includegraphics[width=0.5\textwidth]{figures/controller}
	\caption{Diagramma delle classi del controller.}
	\label{fig:controller}
\end{figure}

\newpage
\section{Test}
\label{sec:test}

Per controllare la qualità di Grigo Verde sono eseguiti diversi test:
\begin{itemize}
	\item Test di unità: per controllare la business logic abbiamo creato una
	      pagina html che invoca delle funzioni di test in php durante la
	      creazione della pagina. Per accedere alla pagina dei test, bisogna
	      rimuovere il commento alla riga 42 del file
	      \texttt{GrigoVerde/src/controller/routes.php}; finalmente è
	      sufficiente accedere alla pagina \texttt{test} a partire dal base url
	      del sito;

	\item Test mediante l'estensione browser Wave: per controllare
	      l'accessibilità in modo automatico;

	\item Test mediante Total Validator: per controllare l'accessibilità in modo
	      automatico;

	\item Test mediante il validatore W3C: per controllare la validità del codice
	      HTML e CSS;

	\item Test manuali: per controllare l'accessibilità in modo manuale;

	\item Test di usabilità: per controllare l'usabilità del sito, anche su
	      dispositivi di diverse dimensioni;

	\item Test di compatibilità: per controllare la compatibilità con i browser
	      più diffusi;

	\item Test sul contrasto dei colori: per controllare che il contrasto dei
	      colori sia sufficiente per garantire l'accessibilità a persone con
	      disabilità visive con Adobe Color.
\end{itemize}

Non abbiamo rilevato errori con i test automatici, mentre con i test manuali
abbiamo riscontrato che il sito è accessibile e usabile.

\newpage
\section{Organizzazione del lavoro}

\subsection{Basso Leonardo}
\begin{itemize}
    \item ...
\end{itemize}

\subsection{Caregnato Simone}
\begin{itemize}
    \item Definizione delle funzionalità;
    \item Definizione del database;
\end{itemize}

\subsection{Igbinedion Osamwonyi Eghosa}
\begin{itemize}
    \item Definizione della mappa concettuale del sito;
\end{itemize}

\subsection{Rosso Carlo}
\begin{itemize}
    \item Coordinamento dei lavori;
    \item Stesura della bozza della relazione;
    \item Definizione dell'infrastruttura di testing;
\end{itemize}


\newpage
\section{Riferimenti alle risorse}

In questa sezione elenchiamo le risorse esterne al corso di laurea utilizzate
per eseguire i test o comunque per la realizzazione del progetto.

\begin{itemize}
	\item \textbf{Github}: per la gestione del codice sorgente e della
	      documentazione, oltre che per collaborare con il team di sviluppo,
	      \url{https://github.com};

	\item \textbf{Excalidraw}: per realizzare dei diagrammi,
	      \url{https://excalidraw.com};

	\item \textbf{Draw.io}: per la realizzazione dei digrammi delle classi
	      seguendo il formalismo UML e per il diagramma del database,
	      \url{https://draw.io};

	\item \textbf{Docker}: per la creazione di container per il testing del
	      progetto, \url{https://www.docker.com};

	\item \textbf{Coolors}: per individuare la palette di colori,
	      \url{https://coolors.co/};

	\item \textbf{Color Adobe}: per controllare il contrasto dei colori,
	      \url{https://color.adobe.com/it/create/color-contrast-analyzer};

	\item \textbf{W3C Markup Validation Service}: per validare il codice
	      HTML, \url{https://validator.w3.org/};

	\item \textbf{W3C CSS Validation Service}: per validare il codice CSS,
	      \url{https://jigsaw.w3.org/css-validator/};

	\item \textbf{Total Validator}: per validare il codice HTML e CSS,
	      \url{https://www.totalvalidator.com/};

	\item Infine sono stati utilizzati plugin all'interno degli editor di
	      testo per controllare staticamente il codice.
\end{itemize}


\end{document}

% \subsection{SEO}
% 
% Innanzitutto, abbiamo immaginao il \textit{search intent} dei nostri utenti.
% Abbiamo identificato le seguenti \textit{query} che potrebbero essere
% utilizzate per cercare il nostro sito:
% \begin{itemize}
% 	\item \textbf{Grigo Verde};
% 	\item \textbf{Liceo Grigoletti};
% 	\item \textbf{Liceo Scientifico Pordenone};
% 	\item \textbf{Spazi verdi};
% 	\item \textbf{Aule all'aperto};
% \end{itemize}
% 
% A partire da queste \textit{query}, abbiamo scritto i tag \texttt{<title>},
% \texttt{<meta name="description">} e \texttt{<meta name="keywords">} delle
% pagine del nostro sito. Si noti, che queste informazioni non sono aggiunte alle
% pagine che richiedono l'autenticazione, in quanto non sono indicizzate dai
% motori di ricerca e non sono accessibili ai visitatori non autenticati. In
% aggiunta, sono state aggiunte keyword specifiche per ogni pagina, in modo da
% migliorare il posizionamento nei motori di ricerca.\\
% Finalmente, sono state adottate soluzioni tecniche, come la
% divisione tra la struttura, la presentazione ed il comportamento per ridurre il
% peso delle pagine e migliorare il \textit{ranking} nei motori di ricerca. Le
% soluzioni tecniche adottate sono spiegate nel dettaglio nelle sezioni dedicate.
