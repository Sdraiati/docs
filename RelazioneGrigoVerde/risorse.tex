\section{Riferimenti alle risorse}

In questa sezione elenchiamo le risorse esterne al corso di laurea utilizzate
per eseguire i test o comunque per la realizzazione del progetto.

\begin{itemize}
	\item \textbf{Github}: per la gestione del codice sorgente e della
	      documentazione, oltre che per collaborare con il team di sviluppo,
	      \url{https://github.com};

	\item \textbf{Excalidraw}: per realizzare dei diagrammi,
	      \url{https://excalidraw.com};

	\item \textbf{Draw.io}: per la realizzazione dei digrammi delle classi
	      seguendo il formalismo UML e per il diagramma del database,
	      \url{https://draw.io};

	\item \textbf{Docker}: per la creazione di container per il testing del
	      progetto, \url{https://www.docker.com};

	\item \textbf{Coolors}: per individuare la palette di colori,
	      \url{https://coolors.co/};

	\item \textbf{Color Adobe}: per controllare il contrasto dei colori,
	      \url{https://color.adobe.com/it/create/color-contrast-analyzer};

	\item \textbf{W3C Markup Validation Service}: per validare il codice
	      HTML, \url{https://validator.w3.org/};

	\item \textbf{W3C CSS Validation Service}: per validare il codice CSS,
	      \url{https://jigsaw.w3.org/css-validator/};

	\item \textbf{Total Validator}: per validare il codice HTML e CSS,
	      \url{https://www.totalvalidator.com/};

	\item Infine sono stati utilizzati plugin all'interno degli editor di
	      testo per controllare staticamente il codice.
\end{itemize}
