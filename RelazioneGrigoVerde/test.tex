\section{Test}
\label{sec:test}

Per controllare la qualità di Grigo Verde sono eseguiti diversi test:
\begin{itemize}
	\item Test di unità: per controllare la business logic abbiamo creato una
	      pagina html che invoca delle funzioni di test in php durante la
	      creazione della pagina. Per accedere alla pagina dei test, bisogna
	      rimuovere il commento alla riga 42 del file
	      \texttt{GrigoVerde/src/controller/routes.php}; finalmente è
	      sufficiente accedere alla pagina \texttt{test} a partire dal base url
	      del sito;

	\item Test mediante l'estensione browser Wave: per controllare
	      l'accessibilità in modo automatico;

	\item Test mediante Total Validator: per controllare l'accessibilità in modo
	      automatico;

	\item Test mediante il validatore W3C: per controllare la validità del codice
	      HTML e CSS;

	\item Test manuali: per controllare l'accessibilità in modo manuale;

	\item Test di usabilità: per controllare l'usabilità del sito, anche su
	      dispositivi di diverse dimensioni;

	\item Test di compatibilità: per controllare la compatibilità con i browser
	      più diffusi;

	\item Test sul contrasto dei colori: per controllare che il contrasto dei
	      colori sia sufficiente per garantire l'accessibilità a persone con
	      disabilità visive con Adobe Color.
\end{itemize}

Non abbiamo rilevato errori con i test automatici, mentre con i test manuali
abbiamo riscontrato che il sito è accessibile e usabile.
