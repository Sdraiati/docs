\section{Test}
\label{sec:test}

Per controllare la qualità di Grigo Verde sono eseguiti diversi test:
\begin{itemize}
	\item Test di unità: per controllare la business logic abbiamo creato una
	      pagina html che invoca delle funzioni di test in php durante la
	      creazione della pagina;

	\item Test mediante l'estensione browser Wave: per controllare
	      l'accessibilità in modo automatico;

	\item Test mediante Total Validator: per controllare l'accessibilità in modo
	      automatico;

	\item Test mediante il validatore W3C: per controllare la validità del codice
	      HTML e CSS;

	\item Test manuali: per controllare l'accessibilità in modo manuale;

	\item Test di usabilità: per controllare l'usabilità del sito, anche su
	      dispositivi di diverse dimensioni;

	\item Test di compatibilità: per controllare la compatibilità con i browser
	      più diffusi.
\end{itemize}

\subsection{Test di accessibilità}

Non abbiamo riscontrato errori con i validatori che abbiamo usato, ma ci sono
alcuni warning che non abbiamo risolto, come per esempio i link ridondanti.
Infatti, per migliorare l'accessibilità, abbiamo deciso di lasciare due link
alla homepage: uno nel logo e uno nella breadcrumb.
