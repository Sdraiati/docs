\section{Progettazione}

\subsection{Design Persona}

Partendo dalla risorsa consigliata nel corso (\textit{Desining for Emotion} di Aaron Walter), abbiamo descritto la personalità di partenza per sviluppare il prodotto:
\begin{itemize}
    \item \textbf{Brand name}: a questo punto il nome del brand è noto, ovvero
    \textit{Grigo Verde};

    \item \textbf{Overview}: 

    \item \textbf{Brand traits}:
        \begin{itemize}
            \item Formale ma non rigido: deve essere adatto ad un contesto scolastico.
            \item Simpatico ma non scherzoso: questo sito deve essere acceduto da dei docenti, non da bambini di quarta elementare.
            \item Schematico ma non rigido: è vero, è un sistema di prenotazioni, ma è al quanto flessibile.
            \item Preciso: sempre pronto a rispondere ad ogni esigenza dell'utente con la massima accuratezza.
        \end{itemize}
    
    \item \textbf{Personality map}:
        \begin{center}
            \begin{tikzpicture}
                \begin{axis}[
                    axis lines = middle,
                    xlabel = { amichevole },
                    ylabel = { formale },
                    xmin = -10, xmax = 10,
                    ymin = -10, ymax = 10,
                    xtick = {-10,-5,...,10},
                    ytick = {-10,-5,...,10},
                ]
            % template per mostrare il punto all'interno del piano cartesiano.
			%		\addplot [
			%			color=black,
			%			mark=*,
			%			only marks,
			%		] coordinates {
			%		(8, -2)
			%	} node[below] {$P$};
                \end{axis}
                
            \end{tikzpicture}
        \end{center}

    
    \item \textbf{Voice}: 
		\begin{itemize}
            \item ...
		\end{itemize}
    
    \item \textbf{Copy examples}: ...
    
    \item \textbf{Visual lexicon}:
		\begin{itemize}
			\item \textbf{Colori}: verde scuro per i titoli, nero per i testi e bianco per lo sfondo;  
			\item \textbf{Contorni}: poco arrotondati: la prenotazione delle aule è una procedura precisa. Inserire dei contorni non 
            arrotondati rafforza l'idea di schematico; 
			\item \textbf{Ombre}:assenti;
			\item \textbf{Font}: \textit{Sans-Serif Arial};
            \item ...
		\end{itemize}
    
    \item \textbf{Engagement methods}:
		\begin{itemize}
			\item \textbf{Feedback}: ad ogni azione corriponde una notifica, in
				modo tale che l'utente abbia sempre un riscontro;

			\item \textbf{Design pulito e intuitivo}: l'utente deve essere in
				grado di capire cosa fare senza dover leggere alcun manuale;

			\item \textbf{Psicologia dei colori}: sono utilizzati dei colori
				accattivanti che richiamano il personaggio scelto;

		\end{itemize}
\end{itemize}

\subsection{Palette}

...

\subsection{Accessibilità}

L'accessibilità è un indice di qualità del sito, pertanto è stata fin da subito
un proposito imprescindibile che ha guidato la fase di progettazione e le
successive. Per quanto siamo incorsi in alcune difficoltà. Di seguito sono
riportate le misure adottate per garantire un'esperienza di utilizzo ottimale
per tutti gli utenti.

\subsubsection{Orientamento dell'utente}

\subsubsection{Colori}

I colori sono stati scelti appositamente con un contrasto elevato tra loro in
modo da garantire una buona leggibilità anche da parte di utenti con deficit
parziale della vista, infatti è stato rispettato lo standard WCAG AA.

\subsubsection{Responsive layout}

Il sito è stato progettato per adattarsi a qualsiasi dispositivo, in modo da
garantire un'esperienza di utilizzo ottimale sia da desktop che da mobile. Per
questo motivo, sono stati adottati layout flessibili e fluidi, in modo da
garantire una buona leggibilità e usabilità indipendentemente dalla dimensione
dello schermo o dalle preferenze dell'utente.

\subsection{Struttura del sito}

Il sito è organizzato in una struttura gerarchica per conciliare facilità di
utilizzo e organizzazione delle informazioni in modo ordinato e preciso. I
contenuti sono suddivisi secondo uno schema organizzativo per argomento di
seguito sono descritte le singole pagine:

\begin{itemize}
    \item ...
\end{itemize}
