\section{Analisi dei requisiti}

\subsection{Target}

\textit{Grigo Verde} è un servizio dedicato alla gestione delle prenotazioni
delle aree verdi della scuola Michelangelo Grigoletti di Pordenone. Il target
principale è costituito dagli insegnanti della scuola, che necessitano di
strumenti semplici per organizzare le lezioni all'aperto.\\
Il servizio è progettato per essere intuitivo e facile da usare, in modo
da permettere agli utenti di prenotare le aree verdi in pochi passaggi. Inoltre,
il servizio vuole essere accessibile da dispositivi mobili, in
modo da permettere agli utenti di prenotare le aree verdi anche in mobilità.
Considerando il target principale, il servizio è progettato per essere
facilmente accessibile e utilizzabile da utenti non esperti di tecnologia.

\subsection{Attori}

Gli attori principali dell'applicazione sono i seguenti:

\begin{itemize}
	\item \textbf{Visitatore}: un utente generico, che non è riconosciuto dal
	      sistema;

	\item \textbf{Utente Docente}: un docente autorizzato, registrato nel
	      sistema e che ha effettuato il login;

	\item \textbf{Utente Amministratore}: un amministratore autorizzato,
	      registrato nel sistema e che ha effettuato il login.
\end{itemize}

\subsection{Funzionalità}

Di seguito sono riportate le funzionalità offerte da \textit{Grigo Verde}:

\begin{enumerate}
	\item \textbf{Autenticazione}: login tramite username e password e quindi
	      riconoscimento dell'utente da parte del sistema; Disponibile a:
	      \begin{itemize}
		      \item Visitatore;
	      \end{itemize}

	\item \textbf{Logout}: Disponibile a:
	      \begin{itemize}
		      \item Utente docente;
		      \item Utente amministratore;
	      \end{itemize}

	\item \textbf{Visualizzazione pagina informativa sull'applicazione}, in cui
	      viene spiegato lo scopo, le motivazioni delle sua creazione e cosa
	      offre;
	      Disponibile a:
	      \begin{itemize}
		      \item Visitatore;
		      \item Utente docente;
		      \item Utente amministratore;
	      \end{itemize}

	\item \textbf{Visualizzazione degli spazi registrati nel sistema}, con foto
	      e nome dello spazio; Disponibile a:
	      \begin{itemize}
		      \item Visitatore;
		      \item Utente docente;
		      \item Utente amministratore;
	      \end{itemize}

	\item \textbf{Filtraggio degli spazi}, per tipo e per disponibilità;
	      Disponibile a:
	      \begin{itemize}
		      \item Visitatore;
		      \item Utente docente;
		      \item Utente amministratore;
	      \end{itemize}

	\item \textbf{Visualizzazione delle dettaglio di uno spazio}:
	      nome, tipo, descrizione, numero di tavoli, immagine, orari di
	      apertura e prenotazioni già effettuate; Disponibile a:
	      \begin{itemize}
		      \item Visitatore;
		      \item Utente docente;
		      \item Utente amministratore;
	      \end{itemize}

	\item \textbf{Inserimento prenotazione}:
	      sono richiesti il giorno e l'orario della prenotazione oltre allo
	      spazio, infine l'utente può inserire una nota opzionale; Disponibile
	      a:
	      \begin{itemize}
		      \item Utente docente;
		      \item Utente amministratore (per poter riservere lo spazio per lavori o altri eventi);
	      \end{itemize}

	\item \textbf{Annullamento prenotazione}:
	      Disponibile a:
	      \begin{itemize}
		      \item Utente docente;
		      \item Utente amministratore;
	      \end{itemize}

	\item \textbf{Modifica prenotazione}: Disponibile a:
	      \begin{itemize}
		      \item Utente docente;
		      \item Utente amministratore;
	      \end{itemize}

	\item \textbf{Modifica delle informazioni personali}: modifica di nome,
	      cognome e password;
	      Disponibile a:
	      \begin{itemize}
		      \item Utente docente;
		      \item Utente amministratore;
	      \end{itemize}

	\item \textbf{Creazione di un nuovo spazio}: disponibile a:
	      \begin{itemize}
		      \item Utente amministratore;
	      \end{itemize}

	\item \textbf{Modifica di uno spazio}: disponibile a:
	      \begin{itemize}
		      \item Utente amministratore;
	      \end{itemize}

	\item \textbf{Eliminazione di uno spazio}: disponibile a:
	      \begin{itemize}
		      \item Utente amministratore;
	      \end{itemize}

	\item \textbf{Modifica degli orari di apertura di uno spazio}: disponibile a:
	      \begin{itemize}
		      \item Utente amministratore;
	      \end{itemize}

	\item \textbf{Visualizzazione della lista degli utenti registrati nel
		      sistema}; Disponibile a:
	      \begin{itemize}
		      \item Utente amministratore;
	      \end{itemize}

	\item \textbf{Registrazione nuovo utente}: aggiunta del nuovo utente al
	      sistema con inserimento di nome, cognome, ruolo (docente o amministratore);
	      Disponibile a:
	      \begin{itemize}
		      \item Utente amministratore;
	      \end{itemize}

	\item \textbf{Eliminazione utente}: disponibile a:
	      \begin{itemize}
		      \item Utente amministratore;
	      \end{itemize}

	\item \textbf{Modifica utente}: modifica di nome, cognome, ruolo e password;
	      Disponibile a:
	      \begin{itemize}
		      \item Utente amministratore;
	      \end{itemize}
\end{enumerate}
