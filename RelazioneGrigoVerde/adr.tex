\section{Analisi dei requisiti}

\subsection{Target}

\textit{Grigo Verde} è un servizio dedicato alla gestione delle prenotazioni
delle aree verdi della scuola Michelangelo Grigoletti di Pordenone. Il target
principale è costituito dagli insegnanti della scuola, che necessitano di
strumenti semplici per organizzare le lezioni all'aperto.\\
Il servizio è progettato per essere intuitivo e facile da usare, in modo
da permettere agli utenti di prenotare le aree verdi in pochi passaggi. Inoltre,
il servizio vuole essere accessibile da dispositivi mobili, in
modo da permettere agli utenti di prenotare le aree verdi anche in mobilità.
Considerando il target principale, il servizio è progettato per essere
facilmente accessibile e utilizzabile da utenti non esperti di tecnologia.

\subsection{Attori}

Gli attori principali dell'applicazione sono i seguenti:

\begin{itemize}
	\item \textbf{Visitatore}: un utente generico, che non è riconosciuto dal
	      sistema;

	\item \textbf{Utente Docente}: un docente autorizzato, registrato nel
	      sistema e che ha effettuato il login;

	\item \textbf{Utente Amministratore}: un amministratore autorizzato,
	      registrato nel sistema e che ha effettuato il login.
\end{itemize}

\subsection{Funzionalità}

\textit{Grigo Verde} offre le seguenti funzionalità:
\begin{enumerate}
	\item \textbf{Autenticazione}: login tramite username e password e quindi
	      riconoscimento dell'utente da parte del sistema; Disponibile a:
	      \begin{itemize}
		      \item Visitatore;
	      \end{itemize}

	\item \textbf{Logout}: Disponibile a:
	      \begin{itemize}
		      \item Utente docente;
		      \item Utente amministratore;
	      \end{itemize}

	\item \textbf{Visualizzazione pagina informativa sull'applicazione}, in cui
	      viene spiegato lo scopo, le motivazioni delle sua creazione e cosa
	      offre;
	      Disponibile a:
	      \begin{itemize}
		      \item Visitatore;
		      \item Utente docente;
		      \item Utente amministratore;
	      \end{itemize}

	\item \textbf{Visualizzazione degli spazi registrati nel sistema}, con foto
	      descrizione, numero di posti; Disponibile a:
	      \begin{itemize}
		      \item Visitatore;
		      \item Utente docente;
		      \item Utente amministratore;
	      \end{itemize}

	\item \textbf{Filtraggio degli spazi per tipo}, ad esempio per aule o per
	      aree ricreative; Disponibile a:
	      \begin{itemize}
		      \item Visitatore;
		      \item Utente docente;
		      \item Utente amministratore;
	      \end{itemize}

	\item \textbf{Filtraggio degli spazi per disponibilità}, tramite selezione
	      del giorno e dell'intervallo temporale; Disponibile a:
	      \begin{itemize}
		      \item Visitatore;
		      \item Utente docente;
		      \item Utente amministratore;
	      \end{itemize}

	\item \textbf{Visualizzazione delle disponibilità orarie di uno spazio}
	      sulla base del mese selezionato; Disponibile a:
	      \begin{itemize}
		      \item Visitatore;
		      \item Utente docente;
		      \item Utente amministratore;
	      \end{itemize}

	\item \textbf{Prenotazione di uno spazio disponibile}, specificando il
	      giorno da calendario e l'orario; Disponibile a:
	      \begin{itemize}
		      \item Utente docente;
	      \end{itemize}

	\item \textbf{Annullamento prenotazione da lui creata in precedenza};
	      Disponibile a:
	      \begin{itemize}
		      \item Utente docente;
	      \end{itemize}

	\item \textbf{Modifica di qualunque prenotazione}: Disponibile a:
	      \begin{itemize}
		      \item Utente amministratore;
	      \end{itemize}

	\item \textbf{Inserimento di prenotazioni periodiche}: inserimento di
	      prenotazioni che si ripetono in un intervallo di tempo; Disponibile a:
	      \begin{itemize}
		      \item Utente amministratore;
	      \end{itemize}

	\item \textbf{Creazione di un nuovo spazio}, con inserimento di immagini,
	      posizione, nome, tipo, descrizione, numero di tavoli; Disponibile a:
	      \begin{itemize}
		      \item Utente amministratore;
	      \end{itemize}

	\item \textbf{Modifica di uno spazio}: inserimento/eliminazione di immagini,
	      modifica posizione, nome, tipo, descrizione, numero di tavoli; Disponibile
	      a:
	      \begin{itemize}
		      \item Utente amministratore;
	      \end{itemize}

	\item \textbf{Eliminazione di uno spazio}: Disponibile a:
	      \begin{itemize}
		      \item Utente amministratore;
	      \end{itemize}

	\item \textbf{Inserimento disponibilità spazio}: inserimento in un
	      calendario settimanale della disponibilità oraria; Disponibile a:
	      \begin{itemize}
		      \item Utente amministratore;
	      \end{itemize}

	\item \textbf{Creazione di un nuovo tipo di spazio}, con specifica del nome
	      e della possibilità o meno di specificare il numero di posti; Disponibile
	      a:
	      \begin{itemize}
		      \item Utente amministratore;
	      \end{itemize}

	\item \textbf{Visualizzazione della lista degli utenti registrati nel
		      sistema}; Disponibile a:
	      \begin{itemize}
		      \item Utente amministratore;
	      \end{itemize}

	\item \textbf{Registrazione nuovo utente}: aggiunta del nuovo utente al
	      sistema con inserimento di nome, cognome, ruolo (docente o amministratore);
	      Disponibile a:
	      \begin{itemize}
		      \item Utente amministratore;
	      \end{itemize}

	\item \textbf{Eliminazione utente}: Disponibile a:
	      \begin{itemize}
		      \item Utente amministratore;
	      \end{itemize}

	\item \textbf{Modifica utente}: modifica di nome, cognome e/o ruolo;
	      Disponibile a:
	      \begin{itemize}
		      \item Utente amministratore;
	      \end{itemize}
\end{enumerate}

\subsection{SEO}

Innanzitutto, abbiamo immaginao il \textit{search intent} dei nostri utenti.
Abbiamo identificato le seguenti \textit{query} che potrebbero essere
utilizzate per cercare il nostro sito:
\begin{itemize}
	\item \textbf{Grigo Verde};
	\item \textbf{Liceo Grigoletti};
	\item \textbf{Liceo Scientifico Pordenone};
	\item \textbf{Spazi verdi};
	\item \textbf{Aule all'aperto};
\end{itemize}

A partire da queste \textit{query}, abbiamo scritto i tag \texttt{<title>},
\texttt{<meta name="description">} e \texttt{<meta name="keywords">} delle
pagine del nostro sito. Si noti, che queste informazioni non sono aggiunte alle
pagine che richiedono l'autenticazione, in quanto non sono indicizzate dai
motori di ricerca e non sono accessibili ai visitatori non autenticati. In
aggiunta, sono state aggiunte keyword specifiche per ogni pagina, in modo da
migliorare il posizionamento nei motori di ricerca.\\
Finalmente, sono state adottate soluzioni tecniche, come la
divisione tra la struttura, la presentazione ed il comportamento per ridurre il
peso delle pagine e migliorare il \textit{ranking} nei motori di ricerca. Le
soluzioni tecniche adottate sono spiegate nel dettaglio nelle sezioni dedicate.
